% \documentclass[type = bachelor]{whu-thesis}
\documentclass[type = master,class = academic]{whu-thesis}
% \documentclass[type = doctor]{whu-thesis}
% type: 学位类型,可选项为 bachelor, master, doctor
% class: 学位类别,可选项为 academic, professional
% showframe: 显示页面布局框架

% 以下仅列举了部分可能用到的设置选项,更多用法请参考文档《whu-thesis.pdf》

% \PassOptionsToPackage{gbnamefmt = lowercase}{biblatex} % 英文作者姓名不强制大写

%%%%%
%注意:latex里的英文封面和授权书和学校最新给的不一致,建议之后用word写这两页再手动加进去。
%%%%%
\whusetup{
  info = {
    title      = {这是一个硕士学术学位\\毕业论文}, % 标题,可使用 \\ 手动换行
    title*     = {abaaba},
    department  = {遥感信息工程学院},
    department* = {School of Remote Sensing and Information Engineering},
    author     = {王亦子},
    author*    = {Wang Yizi},
    student-id = {2023XXXXXX},
    supervisor  = {XXXX},
    supervisor* = {XXXX},
    academic-title  = {教授},
    academic-title* = {Prof},
    % supervisor-outer = {王某某}, % 校外导师(非必填)
    % academic-title-outer = {高级工程师}, % 校外导师职称(非必填)
    % subject = {控制}, % 学科名称(非必填)
    major   = {模式识别与智能系统},
    major*  = {Pattern Recognition and Intelligent System},
    research-area  = {遥感XXXX},
    research-area* = {Remote SensingXXX},
    year = 2026,
    month = 1,
    % clc = , % 分类号
    % udc = ,
    keywords = {遥感, XXXX, XXX, XXX},
    keywords* = {Remote Sensing,  XXXX, XXX, XXX}
  },
  style = {
    % 字体相关选项
    font = termes, % 西文字体,可选项为 default, times, xits, termes
    math-font = termes, % 数学字体,可选项为 default, xits, termes
    cjk-font = mac, % 中文字体,可选项为 windows, mac, fandol(Linux/Overleaf/TexPage), sourcehan, none
    % cjk-fakefont = true, % 使用伪粗体与伪斜体
    % 参考文献及引用相关选项
    bib-backend = bibtex, % 参考文献引擎,可选项为 bibtex, biblatex
    bib-style = numerical, % 参考文献样式,可选项为 numerical, author-year
    % cite-style = <>, % 引用样式(自定义)
    bib-resource = {ref/refs.bib}, % 参考文献数据源
    % 页面相关选项
    % chapter-page-header = true, % 章节首页是否有页眉
    % bachelor-encover = true, % 本科毕业论文英文封面
    library, % 图书馆模式(去掉论文中所有的空白页)
    license = false % 使用授权协议书(这个协议书和学校新给的长的不一样,建议自己打出来加进去)
    % fullwidth-stop = true, % 句号样式
    % footnote-style = <>, % 脚注编号样式
    % abstract-keywords-type  = blankline, % 摘要与关键词之间样式,可选项为 blankline, newline, vfill
    % abstract-keywords-type* = blankline, % 摘要与关键词之间样式,可选项为 blankline, newline, vfill
  }
}
\whumodule{algorithm2e}
\begin{document}

\tableofcontents % 目录
% \listoffigures % 图目录
% \listoftables % 表目录

% 符号表
% \begin{notation}
%   $\omega_n$ & $n$-维欧氏空间中单位球的表面积 \\
%   $\alpha_n$ & $n$-维欧氏空间中单位球的体积 \\
% \end{notation}

\mainmatter

% 当然你也可以直接在这里写,不过这样不太方便管理
\chapter{绪论}

\section{研究背景与意义}
这是很有背景的研究,这是引用一篇文章\cite{JSGG202119006}。
% 段落要空一行
这个研究很有意义。

这个研究非常有意义。

\section{国内外研究现状}
\subsection{基于XXXX的XXXX}

XXXX旨在XXXX,其应用范围广泛。

随着XXXX的兴起,XXXX取得了突破。

\subsection{基于XXXX的XXXX}
随着XXXX的兴起,XXXX取得了突破。

\subsection{研究现状小结}

综上所述,XXX。

\section{研究内容与章节安排}
\subsection{主要研究内容}
本研究XXXX

\subsection{技术路线}
本研究的技术路线:

\subsection{章节安排}
本文共分为五个章节,具体安排如下:

第一章:绪论。本章XXX

第二章:XXXX理论及相关算法。XXX

第三章:XXX的XXX方法。XXXX

第四章:XXXX

第五章:总结与展望。
\include{pages/chapter2.tex}
\chapter{基于XXXXX方法}
本节详细介绍本文提出的XXXXX。

\section{整体架构}
XXXXX

如图~\ref{fig:model}所示。
\begin{figure}[ht]
    \centering
    \includegraphics[width = 10cm]{whu-logo.png}
    \caption{XXXX}
    \label{fig:model}
\end{figure}

XXXXX

\section{XXXXX}
XXXXX


\section{本章小结}
XXXXX








\chapter{实验与结果分析}
在本节中,XXXXX

\section{实验设置}

XXXXX

交并比(IoU)定义如下:

\begin{equation}
\text{IoU} = \frac{\text{TP}}{\text{TP} + \text{FP} + \text{FN}}
\end{equation}


\section{指标对比}
表~\ref{table:simple}表明XXX
\begin{table}[ht]
  \centering
  \caption{%
    简单的表格和引用 abc 123 %\cite{whu-bachelor:1}
  }
  \label{table:simple}
  \begin{tabular}{cc}
    \hline
    a & b \\ \hline
    c & d \\ \hline
    测试 & 文本 \\ \hline
  \end{tabular}
\end{table}


\section{消融实验}

实验结果证明:

\begin{itemize}
\item 这个模块很有用。
\item 这个模块也很有用。
\end{itemize}


\section{本章小结}
XXXXXX
\chapter{总结与展望}
本章XXXX

\section{总结}
XXXXX

\section{展望}
XXX


% 参考文献
% \nocite{*}
\printbibliography

% 附录
\appendix

% 附录
\chapter{读硕期间发表的科研成果目录}
1. 小论文

2. 专利

\chapter{读硕期间参与的课题情况}
1. 比赛

2. 课题



% 致谢
\begin{acknowledgements}
  致谢是以简短的文字对课题研究与论文撰写过程中直接给予帮助的人员(例如指导教师、答疑教师及其他人员)表示谢意。致谢是对他人劳动的尊重,也是学术规范。内容限一页。
\end{acknowledgements}



\end{document}